\documentclass[a4paper]{article}

%%%%%%%%%%%%%
% Preambula %
%%%%%%%%%%%%%

% Kodiranje
\usepackage[utf8]{inputenc}
\usepackage[T1]{fontenc}
\usepackage[slovene]{babel}

\usepackage{textcase}

\usepackage{lmodern}

\usepackage{hyperref}

% Matematika
\usepackage{amsmath}  % razna okolja za poravnane enačbe ipd.
\usepackage{amsthm}   % definicije okolij za izreke, definicije, ...
\usepackage{amssymb}  % dodatni matematični simboli
\usepackage{xypic}    % paket za diagrame

% Definicija okolij izrek, posledica
{\theoremstyle{theorem}
  \newtheorem{izrek}{Izrek}[section]
  \newtheorem{posledica}[izrek]{Posledica}
}

% Definicija okolja aksiom
{\theoremstyle{definition}
  \newtheorem{aksiom}{Aksiom}[section]
}

% Definicija okolij za definicije in vaje
{\theoremstyle{definition}
  \newtheorem{definicija}[izrek]{Definicija}
  \newtheorem{vaja}[izrek]{Vaja}
  \newtheorem{primer}[izrek]{Primer}
}

\title{Analiza 1. Predavanja}
\author{Matija Sirk}

% Definicije makrov
\newcommand{\nn}{\mathbb{N}} % simbol za naravna števila
\newcommand{\zz}{\mathbb{Z}} % simbol za cela števila
\newcommand{\qq}{\mathbb{Q}} % simbol za racionalna števila
\newcommand{\rr}{\mathbb{R}} % simbol za realna števila
\newcommand{\cc}{\mathbb{C}} % simbol za kompleksna števila
\newcommand{\konecdokaza}{$\blacksquare$} % črn (namesto prazen) kvadratek za konec dokaza

%%%%%%%%%%%
% Vsebina %
%%%%%%%%%%%

\begin{document}

\maketitle

\begin{abstract}
  Skupni zapiski s predavanj Analize 1.

  Lose all hope you who enter here.
\end{abstract}



\newpage
\tableofcontents
\newpage

\section{Števila}

\subsection{Naravna Števila}
So števila s katerimi štejemo.
Množica naravnih števil: $\nn = \{1, 2, 3, \dots\}$

Vsako naravno število $n$ ima svojega naslednika $n^+$.
$1$ je edini element $\nn$, ki ni naslednik nobenega $n \in \nn$.
 
Opisana so s Peanovimi aksiomi. Naravna števila so množica, skupaj s pravilom, ki vsakemu naravnemu številu $n$ priredi njegovega naslednika $n^+$, ki je prav tako naravno število in velja:

\begin{aksiom}
   $\forall m, n \in \nn :  m^+ = n^+ \Leftrightarrow m = n$
\end{aksiom}

\begin{aksiom}
  $Obstaja ~1 \in \nn, ki ~ni ~n^+ ~nobenega ~n \in \nn$
\end{aksiom}

\begin{aksiom}
  $A \subset \nn ~in ~1 \in A ~in ~n \in A \Rightarrow n^+ \in A \Leftrightarrow A = \nn$
\end{aksiom}

Tretjemu Peanovemu aksiomu pravimo tudi aksiom popolne indukcije.
Pazi, da je tukaj simbol $\subset$ simbol za podmnožico, ne za pravo podmnožico.

\begin{primer}
  Dokaži, da je za vsak $n \in \nn$ izraz $4^n - 3n + 8$ deljiv z $9$.

  \begin{gather*}
    n = 1: \\
    4^1 - 3*1 + 8 = 9,~9 \mid 9~ \texttt{(Dokaz za bazo indukcije)}\\
    \\
    n \rightarrow n+1: \\
    4^n - 3n + 8 = 9k,~k \in \zz \\
    4^n = 9k + 3n - 8~ \texttt{(Indukcijska predpostavka)}\\
    \\
    4^{n+1} - 3(n+1) + 8 = 4*4^n - 3n + 5 = 4*(9k + 3n - 8) - 3n + 5 =\\
    = 9*4k + 9n - 9*3 = 9(4k + n - 3), \\
    9 \mid 9(4k + n - 3)~ \texttt{(Dokaz za indukcijski korak)}
  \end{gather*}

  Ker velja trditev za bazo indukcije in ker za vse $n \in \nn$ velja indukcijski korak, po načelu
  popolne indukcije trditev velja $\forall n \in \nn$. \konecdokaza
\end{primer}

Naravna števila lahko seštevamo in množimo (aksiomatska definicija bo pri \emph{Logiki in Množicah}).
Prav tako jih lahko uredimo po velikosti - $\nn$ je (linearno) urejena. Vsaka neprazna podmnožica $\nn$
ima najmanši element in je urejena. Vsaka končna neprazna podmnožica $\nn$ ima največji element.

\subsection{Cela Števila}

V celih številih je smiselno definirano tudi odštevanje. $\zz = \{\dots, -3, -2, -1, 0, 1, 2, 3, \dots\}$

Seštevanje in množenje se z $\nn$ razširita tudi na $\zz$. Tudi $\zz$ ima definirano urejenost na
običajen način.

Urejenost:
\begin{gather*}
  x - y = 0 \Rightarrow x = y \\
  x - y > 0 \Rightarrow x > y
\end{gather*}

V splošnem deljenje ni definirano.

\subsection{Racionalna Števila}

Racionalna števila so kvocienti celih in naravnih števil. $\qq = \{\frac{m}{n};~ m \in \zz,~ n \in \nn\}$

Racionalna števila lahko pogosto napišemo na več načinov - različni ulomki lahko zaznamujejo isto število (kvocient):
$\frac{4}{3} = \frac{8}{6}$

Enakost v racionalnih številih:
$\frac{m}{n} = \frac{k}{l} \Leftrightarrow ml = kn$

Racionalna števila lahko naivno skonstruiramo takole:

\begin{equation}
  \zz \times \nn = \{(m,n);~ m \in \zz,~ n \in \nn\}
\end{equation}

To množico razdelimo na ekvivalenčne razrede tako, da grupiramo skupaj v razred tiste ulomke, ki
predstavljajo isti kvocient. Torej, urejena para $\frac{m}{n}$ in $\frac{k}{l}$ sta v istem razredu, če zanju velja:
$\frac{m}{n} = \frac{k}{l} \Leftrightarrow ml = kn$.
Racionalno število je razred urejenih parov in ga označimo z $\frac{m}{n}$ (oziroma z poljubnim drugim predstavnikom
razreda).

Računske operacije v $\qq$:
\end{document}
